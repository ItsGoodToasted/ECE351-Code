\documentclass[11pt]{exam}
\newcommand{\myname}{Nigel Lee}
\newcommand{\myemail}{Lee3188@vandals.uidaho.edu}
\newcommand{\myhwtype}{Pre-Lab}
\newcommand{\myhwnum}{5}
\newcommand{\myclass}{ECE-351}
\newcommand{\mylecture}{}
\newcommand{\mysection}{}

% Prefix for numedquestion's
\newcommand{\questiontype}{Question}

% Use this if your "written" questions are all under one section
% For example, if the homework handout has Section 5: Written Questions
% and all questions are 5.1, 5.2, 5.3, etc. set this to 5
% Use for 0 no prefix. Redefine as needed per-question.
\newcommand{\writtensection}{0}

\usepackage{amsmath, amsfonts, amsthm, amssymb, mathtools}  % Some math symbols
\usepackage{enumerate}
\usepackage{graphicx}
\usepackage{hyperref}
\usepackage[all]{xy}
\usepackage{wrapfig}
\usepackage{fancyvrb}
\usepackage[T1]{fontenc}
\usepackage{listings}
\usepackage{centernot}
\usepackage{sectsty,ulem}  % section styling, used for the coloration of headers
\usepackage{subfig}   % used to create figures of sub-figures
\usepackage{float}    % Package for improving figure locations
\usepackage{xcolor} % allows us to use HTML color formats
\usepackage{cite}
\usepackage{pdfpages} % used for importing PDFs as their own pages \includepdf
\usepackage{appendix} % appendix stuff: https://texfaq.org/FAQ-appendix

%% Code block stuff
\definecolor{codegreen}{rgb}{0,0.6,0}
\definecolor{codegray}{rgb}{0.5,0.5,0.5}
\definecolor{codepurple}{rgb}{0.58,0,0.82}
\definecolor{backcolour}{rgb}{0.98,0.98,0.98}

\lstdefinestyle{mystyle}{
    backgroundcolor=\color{backcolour},
    commentstyle=\color{codegreen},
    keywordstyle=\color{magenta},
    numberstyle=\tiny\color{codegray},
    stringstyle=\color{codepurple},
    basicstyle=\ttfamily\footnotesize,
    breakatwhitespace=true,
    breaklines=true,
    captionpos=b,
    keepspaces=true,
    numbers=left,
    numbersep=4pt,
    showspaces=false,
    showstringspaces=false,
    showtabs=true,
    tabsize=4
}
\lstset{style=mystyle}
%% end code block stuff

\linespread{1.3} % Line spacing

\DeclarePairedDelimiter{\ceil}{\lceil}{\rceil}
\DeclarePairedDelimiter{\floor}{\lfloor}{\rfloor}
\DeclarePairedDelimiter{\card}{\vert}{\vert}

% Uncomment the following line to get Solarized-themed source listings
% You will have had to already installed the solarized-light package
% https://github.com/jez/latex-solarized
%
%\usepackage{solarized-light}

\setlength{\parindent}{0pt}
\setlength{\parskip}{5pt plus 1pt}
\pagestyle{empty}

\def\indented#1{\list{}{}\item[]}
\let\indented=\endlist

\newcounter{questionCounter}
\newcounter{partCounter}[questionCounter]

\newenvironment{namedquestion}[1]{
    \addtocounter{questionCounter}{1}%
    \setcounter{partCounter}{0}%
    \vspace{.2in}%
        \noindent{\bf #1}%
    \vspace{0.3em} \hrule \vspace{.1in}%
}{}

\newenvironment{numedquestion}[0]{%
	\stepcounter{questionCounter}%
    \vspace{.2in}%
        \ifx\writtensection\undefined
        \noindent{\bf \questiontype \; \arabic{questionCounter}. }%
        \else
          \if\writtensection0
          \noindent{\bf \questiontype \; \arabic{questionCounter}. }%
          \else
          \noindent{\bf \questiontype \; \writtensection.\arabic{questionCounter} }%
        \fi
    \vspace{0.3em} \hrule \vspace{.1in}%
}{}

\newenvironment{alphaparts}[0]{%
  \begin{enumerate}[label=\textbf{\Alph*.}]
}{\end{enumerate}}

\newenvironment{arabicparts}[0]{%
  \begin{enumerate}[label=\textbf{\arabic{questionCounter}.\arabic*})]
}{\end{enumerate}}

\newenvironment{questionpart}[0]{%
  \item
}{}

\newcommand{\answerbox}[1]{
\begin{framed}
\vspace{#1}
\end{framed}}

\pagestyle{head}

\headrule
\header{\textbf{\myclass\ \mylecture\mysection}}%
{\textbf{\myname\ (\myemail)}}%
{\textbf{\myhwtype\ \myhwnum}}

\begin{document}
\thispagestyle{plain}
\begin{center}
  {\Large \myclass{} \myhwtype{} \myhwnum} \\
  \myname{} (\myemail{}) \\
  Febuary 18, 2025
\end{center}

\begin{namedquestion}{Task 1: Find the transfer function}
    \begin{flalign*}
    & V_{out} = V_{in} \cdot \frac{Z_{LC}}{Z_R + Z_{LC}} : Z_{LC} = 
        (\frac{1}{Z_L}+\frac{1}{Z_C})^{-1} = 
        (\frac{1}{j\omega L}+j\omega C)^{-1} =
        \frac{1}{\frac{1}{j \omega L}+j \omega C}=
        \frac{j \omega L}{1 - \omega^{2}LC}& \\
    & \Rightarrow V_{out} = V_{in} \frac{\frac{j \omega L}{1 - \omega^{2}
        LC}}{R + \frac{j \omega L}{1 - \omega^{2}LC}}
        = V_{in}\frac{j \omega L}{R(1-\omega^2LC) + j \omega L}
        : s = j \omega \quad s^2 = -\omega^2 & \\
    & \rightarrow \frac{V_{out}}{V_{in}} = \frac{sL}{s^2RLC+sL+R}
        = \frac{s\frac{1}{RC}}{s^2+s\frac{1}{RC}+\frac{1}{LC}}  &
    \end{flalign*}
\end{namedquestion}

\begin{namedquestion}{Task 2: Find the impulse response}
    \begin{flalign*}
    & p = \dfrac{-\dfrac{1}{RC}+\sqrt{(\dfrac{1}{RC})^2-4\dfrac{1}{LC}}}{2}
        = -\frac{1}{2RC}+\frac{1}{2}\sqrt{\frac{1}{R^2C^2}-\frac{4}{LC}}
        : \alpha = -\frac{1}{2RC} \quad \beta=\frac{1}{2}\sqrt{\frac{1}{R^2C^2}-\frac{4}{LC}}& \\[1em]
    & g = \left.s\frac{1}{RC}\right|_{s=p} =
    \frac{-1}{2R^2C^2}+\sqrt{\frac{1}{4R^4C^4}-\frac{1}{R^2C^3L}} & \\[1em]
    & |g| = \sqrt{\frac{1}{4R^4C^4} +\frac{1}{4R^4C^4} -\frac{1}{R^2C^3L}}
    = \sqrt{\frac{1}{2R^4C^4}-\frac{1}{R^2C^3L}} & \\[1em]
    & \angle g = \tan^{-1}\left(\dfrac{\sqrt{\dfrac{1}{4R^4C^4}-
        \dfrac{1}{R^2C^3L}}}{\dfrac{-1}{2R^2C^2}}\right)
        = \tan^{-1}\left(-\sqrt{\frac{4R^4C^4}{4R^4C^4}-\frac{4R^2C^4}{R^2C^3L}} \right) 
        =\tan^{-1}\left(-\sqrt{1-\frac{4R^2C}{L}}\right) & \\[1em]
    &h(t) = \frac{\sqrt{\dfrac{1}{2R^4C^4}-\dfrac{1}{R^2C^3L}}}{-j\dfrac{1}{2}\sqrt{\dfrac{1}{R^2C^2}-\dfrac{4}{LC}}}exp\left(\dfrac{-1}{2RC}t\right)
    \sin\left[-j*\frac{1}{2}\sqrt{\frac{1}{R^2C^2}-\frac{4}{LC}}t
    +\tan^{-1}\left(-\sqrt{1-\frac{4R^2C}{L}}\right)\right]u(t) & \\[1em]
    &h(t) = -j\sqrt{\dfrac{\dfrac{1}{2R^4C^4}-\dfrac{1}{R^2C^3L}}{\dfrac{1}{4R^2C^2}-\dfrac{1}{LC}}}exp\left(\frac{-1}{2RC}t\right)
    \sin\left[-j*\frac{1}{2}\sqrt{\frac{1}{R^2C^2}-\frac{4}{LC}}t
    +\tan^{-1}\left(-\sqrt{1-\frac{4R^2C}{L}}\right)\right]u(t) &
    \end{flalign*}
    Note: the -j term is to seperate the j component of $p = \alpha +j \omega$
\end{namedquestion}

\end{document}